\documentclass[12pt,letterpaper]{article}
\usepackage[empty]{fullpage}
\usepackage[utf8]{inputenc}
\usepackage[spanish]{babel}
\usepackage{amsmath}
\usepackage{amsfonts}
\usepackage{amssymb}
\usepackage{graphicx}
\usepackage{geometry}
\usepackage{hyperref}
\usepackage{listings} % Paquete para mostrar código
\usepackage{xcolor}   % Paquete necesario para usar \color
\usepackage{fvextra}
\usepackage{adjustbox}

%-----------------------------------------------------------------------------------------
% Definir el entorno "exercise"
\newtheorem{exercise}{Ejercicio}
\geometry{lmargin=1.5cm,rmargin=1.5cm,tmargin=0.5cm,bmargin=0.5cm}
\lstdefinestyle{bashstyle}{ 
  language=bash,                % El lenguaje es Bash
%  basicstyle=\ttfamily,         % Estilo básico de fuente
%  keywordstyle=\color{blue},    % Color para palabras clave
%  commentstyle=\color{gray},    % Color para comentarios
%  showstringspaces=false        % No mostrar espacios en cadenas
  	backgroundcolor=\color{black},   
	commentstyle=\color{gray},
	keywordstyle=\color{green},
   numberstyle=\tiny\color{gray},
   stringstyle=\color{white},
   basicstyle=\ttfamily\footnotesize,
   breakatwhitespace=false,         
   breaklines=true,                 
   captionpos=b,                    
   keepspaces=true,                 
   numbers=left,                    
   numbersep=5pt,                  
   showspaces=false,                
   showstringspaces=false,
   showtabs=false,                  
   tabsize=2
}
\hypersetup{colorlinks=true, linkcolor=blue}
%-------------------------------------------------------------------------------------------------------------
\begin{document}
%\maketitle
\begin{minipage}[c]{4.8cm}
	\vspace{6mm}
	\includegraphics[width=4.8cm, height=3.2cm]{images/Logo_de_la_Escuela_Nacional_de_Estudios_Superiores_Unidad_Mérida.png}
\end{minipage}
\begin{minipage}[c]{12cm}
	\vspace{6mm}
	\center
	\huge{Taller de Investigación en Ciencias Ambientales I} \\ \bigskip
	\Large{Tarea \# 1} \\
	\large{Interfaz de Línea de Comandos (Terminal)}\\ \bigskip
	\today
\end{minipage}
%-------------------------------------------------------------------------------------------------------------
\vspace{3mm}
\hrule
\vspace{3mm}
Para cada uno de los siguientes ejercicios tendrás que tomar una captura a tu interfaz de línea de comandos (terminal) donde se reflejen las actividades solicitadas. La fecha máxima de entrega es el miércoles 28 de agosto de 2024. 
\begin{exercise}
%\begin{lstlisting}[style=bashstyle]
%#!/bin/bash
%echo "Hola, Mundo"
%for i in {1..5}
%do
%echo "Numero $i"
%done
%\end{lstlisting}
Descarga el repositorio del curso: \href{https://github.com/ENES-Merida/taller-de-investigacion-en-ciencias-ambientales-I/archive/refs/heads/main.zip}{click aquí} y descomprímelo. Usando la terminal de Windows mueve la carpeta \textbf{taller-de-investigacion-en-ciencias-ambientales-I} hacia el directorio \verb|C:\Users\tu_nombre_de_usuario|. Entra a la carpeta \textbf{taller-de-investigacion-en-ciencias-\\ambientales-I} y muestra su contenido en la terminal usando el comando \verb|tree|.
\end{exercise}
%-------------------------------------------------------------------------------------------------------------
\begin{exercise}
Desde la terminal de Windows entra a la carpeta \textbf{taller-de-investigacion-en-ciencias-\\ambientales-I} y crea las carpetas \textbf{bibliografia} y \textbf{tareas\textbackslash tarea\_1}. Descarga los libros que se te proporcionaron en el Classroom de la materia y usando la terminal de Windows muévelos a la carpeta bibliografia. Muestra el contenido de las carpetas \textbf{bibliografia} y \textbf{tareas} usando dos maneras distintas.
\end{exercise}
%-------------------------------------------------------------------------------------------------------------
\begin{exercise}
Elimina la carpeta \textbf{resources} y el archivo \textbf{README.md}, que se encuentran dentro del directorio \verb|C:\Users\tu_nombre_de_usuario\taller-de-investigacion-en-ciencias-|\\\verb|ambientales-I|, usando la terminal de Windows. Muestra el contenido de la carpeta \textbf{taller-de-\\investigacion-en-ciencias-ambientales-I} con el comando de tu elección.
\end{exercise}
%-------------------------------------------------------------------------------------------------------------
\begin{exercise}
Usando el comando apropiado crea un archivo al que llamarás \textbf{creado\_por\_tuNombre} de extensión \verb|txt| y guárdalo en la carpeta \textbf{tarea\_1}. Agrega, sin usar algún editor de texto, el siguiente texto al archivo que has creado: \textbf{Cuando despertó, el dinosaurio seguía ahí}. Visualiza en la terminal el contenido del archivo creado.
\end{exercise}
%-------------------------------------------------------------------------------------------------------------
\begin{exercise}
Con el comando \verb|history| podemos ver el historial de todas los comandos que hemos ejecutado en la terminal desde que la abrimos. Usando la terminal de Windows guarda su contenido en un archivo de extensión \verb|txt| que nombrarás como \textbf{historial\_de\_comandos\_tuNombre}. Este archivo deberás guardarlo en la carpeta \textbf{tarea\_1}. Visualiza el contenido del archivo con el comando apropiado.
\end{exercise}
\end{document}
%-------------------------------------------------------------------------------------------------------------